\documentclass[a4paper,12pt]{article}


% add more packages if necessary
\usepackage{xspace}
%\usepackage{graphicx}
%\usepackage{xcolor}
%\usepackage{hyperref}


\newcommand{\groupname}{Panda\xspace}


\title{
Project Report \\ 
Group Panda \\
\vspace{5mm}
\large Java and C\# in depth, Spring 2013
}
\author{
Christian Klauser \\
Sander Schaffner \\
Roger Walt
}
\date{\today}



\begin{document}
\maketitle

\section{Introduction}

This document describes the design and implementation of the \emph{Panda Virtual File System} of group \emph{Panda}. The project is part of the course \emph{Java and C\# in depth} at ETH Zurich. The following sections describe each project phase, listing the requirements that were implemented and the design decisions taken. The last section describes a use case of using the \emph{Panda Virtual File System}.

% PART I: VFS CORE
% --------------------------------------

\section{VFS Core}

The VFS Core is the system part of the virtual file system. It manages directories and files as we are used to it in other systems like linux.\\

We introduced a block system. Each block can be specified in one of 4 different types. To achieve code which can be good maintained we designed 3 layers. These block types and layers are described in the Design section.

\subsection{Requirements}

% TODO: Remove this text and replace it with actual content
\emph{Describe which requirements (and possibly bonus requirements) you have implemented in this part. Give a quick description (1-2 sentences) of each requirement. List the software elements (classes and or functions) that are mainly involved in implementing each requirement.}

\begin{enumerate}
	\item \emph{The virtual disk must be stored in a single file in the working directory in the host file system.}\\
		TODO
	\item \emph{VFS must support the creation of a new disk with the specified maximum size at the specified
location in the host file system.}\\
		TODO
	\item \emph{VFS must support several virtual disks in the host file system.}\\
		TODO
	\item \emph{VFS must support disposing of the virtual disk.}\\
		TODO
	\item \emph{VFS must support creating/deleting/renaming directories and files.}\\
		TODO
	\item \emph{VFS must support navigation: listing of files and folders, and going to a location expressed by a
concrete path.}\\
		TODO
	\item \emph{VFS must support moving/copying directories and files, including hierarchy.}\\
		TODO
	\item \emph{VFS must support importing files and directories from the host file system.}\\
		TODO
	\item \emph{VFS must support exporting files and directories to the host file system.}\\
		TODO
	\item \emph{VFS must support querying of free/occupied space in the virtual disk.}\\
		TODO
	\item \textbf{Bonus Basic:} \emph{Elastic disk: Virtual disk can dynamically grow or shrink, depending on its occupied space.}\\
		The intern data structure can already grow and shrink dynamically.
	\item \textbf{Bonus Advanced:} \emph{Large data: This means, that VFS core can store \& operate amount of data, that can't fit to PC
RAM ( typically, more than 4Gb).}\\
		To fullfill this taks we decided to work with a block structure.
\end{enumerate}

\subsection{Design}

% TODO: Remove this text and replace it with actual content
\emph{Give an overview of the design of this part and describe in general terms how the implementation works. You can mention design patterns used, class diagrams, definition of custom file formats, network protocols, or anything else that helps understand the implementation.}

\subsection{Specification}

This section lists the detailed specification of the virtual file system.

\subsubsection{General Remarks}

\begin{itemize}
	\item Offsets \& lengths in bytes.
	\item The VFS is organized in blocks with fixed BLOCK\_SIZE.
	\item All addresses are in number of blocks from 0 and of length 4 bytes.
	\item Only single links to blocks (not more than one hard-link) are allowed. This means that one file or directory can only be in one directory.
	\item Block address 0 is illegal, it means absence of a block.
	\item B\_S := BLOCK\_SIZE \& d-t := data-type
	\item Offsets are absolute
	\item Strings are encoded in UTF-8
\end{itemize}

\subsubsection{Metadata}

Metadata of the whole VFS starts at address 0.\\

\begin{tabular}{|p{1.5cm}|p{1.5cm}|p{1.5cm}|p{7cm}|}\hline
Offset 	&Length	&C\# d-t	&Description\\\hline
0	&4	&UInt32	&Number of blocks in entire VFS\\
4	&4	&UInt32	&BLOCK\_SIZE in bytes\\
8	&4	&UInt32	&Address of root directory node\\
12	&4	&UInt32	&Address of empty page block. Must never be 0\\
16	&4	&UInt32	&“break” in number of blocks, see empty space management.\\
20	&B\_S -20	&UInt32	&Empty (initialized with 0)\\\hline
\end{tabular}

Normal blocks are everywhere but at address 0.

\paragraph{Block Types}

\begin{itemize}
	\item Directory blocks (many different blocks, with optional continuation blocks)
	\item File blocks (many different blocks, with optional continuation blocks)
	\item Data blocks (many different blocks)
	\item Empty space block (exactly one block, with optional continuation blocks)
\end{itemize}

\subparagraph{Directory blocks} \mbox{} \\

Contain file / directory names of current directory and their block addresses.\\

\begin{tabular}{|p{1.5cm}|p{1.5cm}|p{1.5cm}|p{7cm}|}\hline
Offset 	&Length	&C\# d-t	&Description\\\hline
0	&?	&-	&Arbitrary number of directory entries\\
B\_S - 4	&4	&UInt32	&Link to directory continuation block. 0 here marks absence of continuation blocks.\\\hline
\end{tabular} \\

Directory continuation blocks look the same as directory blocks and can link to other directory continuation blocks.

\subparagraph{Directory entry} \mbox{} \\

\begin{tabular}{|p{1.5cm}|p{1.5cm}|p{1.5cm}|p{7cm}|}\hline
Offset 	&Length	&C\# d-t	&Description\\\hline
0	&1	&UInt8	&If first bit (the least significant) set (== 1), following address points to directory. Else to file.\\
2	&1	&UInt8	&Number of bytes in file name. 0 here marks end of directory block.\\
3	&X	&String	&File / directory name\\
X	&X + 4	&UInt32	&Address to file / directory block\\\hline
\end{tabular} \\

\subparagraph{File blocks} \mbox{} \\

Contain addresses to data blocks.\\

\begin{tabular}{|p{1.5cm}|p{1.5cm}|p{1.5cm}|p{7cm}|}\hline
Offset 	&Length	&C\# d-t	&Description\\\hline
0	&8	&UInt64	&File size in bytes (to manage files smaller than block size)\\
8	&?	&UInt32	&Arbitrary number of addresses to data blocks\\
B\_S - 4	&4	&UInt32	&Link to file continuation block. 0 here marks absence of continuation blocks.\\\hline
\end{tabular} \\

File continuation blocks have file size 0 and can link to other file continuation blocks.

\subparagraph{File blocks} \mbox{} \\

Contain only plain binary data.

\subparagraph{Empty space block} \mbox{} \\

Contains addresses to empty blocks.\\

\begin{tabular}{|p{1.5cm}|p{1.5cm}|p{1.5cm}|p{7cm}|}\hline
Offset 	&Length	&C\# d-t	&Description\\\hline
0	&4	&UInt32	&Number of empty blocks in number of blocks\\
4	&?	&UInt32	&Arbitrary number of addresses to empty blocks\\
BLOCK\_SIZE ? 4	&4	&UInt32	&Link to empty space continuation block. 0 here marks absence of continuation blocks.\\\hline
\end{tabular} \\

\subsubsection{Empty space management}

The VFS is designed to maintain an index of unused blocks. The addresses of the unused blocks are stored in the empty space block. Its address is stored in the VFS meta-data. This empty space block may also have empty space continuation blocks. But not every address to an empty block in the whole VFS can be stored in this empty space block. Instead, only addresses of empty blocks up to a maximum address, which is called “break”, is stored in this block. If there are no empty blocks left, the “break” must be increased by 1, and the new empty block addresses must be added to the empty space block. If the block next to “break” is freed, decrease the “break”, otherwise the address of this block to the empty space block or its last continuation block.

% PART II: VFS Browser
% --------------------------------------

\section{VFS Browser}

% TODO: Remove this line
\textbf{[This section has to be completed by April 22nd.]}

%TODO: Remove this text and replace it with actual content
\emph{Give a short (1-2 paragraphs) description of what VFS Browser is.}


\subsection{Requirements}

% TODO: Remove this text and replace it with actual content
\emph{Describe which requirements (and possibly bonus requirements) you have implemented in this part. Give a quick description (1-2 sentences) of each requirement. List the software elements (classes and or functions) that are mainly involved in implementing each requirement.}


\subsection{Design}

% TODO: Remove this text and replace it with actual content
\emph{Give an overview of the design of this part and describe in general terms how the implementation works. You can mention design patterns used, class diagrams, definition of custom file formats, network protocols, or anything else that helps understand the implementation.}


\subsection{Integration}

% TODO: Remove this text and replace it with actual content
\emph{If you had to change the design or API of the previous part, describe the changes and the reasons for each change here.}



% PART III: Synchronization Server
% --------------------------------------

\section{Synchronization Server}

% TODO: Remove this line
\textbf{[This section has to be completed by May 13th.]}

%TODO: Remove this text and replace it with actual content
\emph{Give a short (1-2 paragraphs) description of what VFS Browser is.}


\subsection{Requirements}

% TODO: Remove this text and replace it with actual content
\emph{Describe which requirements (and possibly bonus requirements) you have implemented in this part. Give a quick description (1-2 sentences) of each requirement. List the software elements (classes and or functions) that are mainly involved in implementing each requirement.}


\subsection{Design}

% TODO: Remove this text and replace it with actual content
\emph{Give an overview of the design of this part and describe in general terms how the implementation works. You can mention design patterns used, class diagrams, definition of custom file formats, network protocols, or anything else that helps understand the implementation.}


\subsection{Integration}

% TODO: Remove this text and replace it with actual content
\emph{If you had to change the design or API of the previous part, describe the changes and the reasons for each change here.}



% PART IV: Quick Start Guide
% --------------------------------------

\section{Quick Start Guide}

% TODO: Remove this line
\textbf{[optional: This part has to be completed by April 8th.]}

% TODO: Remove this text and replace it with actual content
\emph{If you have a command line interface for your VFS, describe here the commands available (e.g. ls, copy, import).} \\ \\ \\


% TODO: Remove this line
\noindent\textbf{[This part has to be completed by May 13th.]}

% TODO: Remove this text and replace it with actual content
\emph{Describe how to realize the following use case with your system. Describe the steps involved and how to perform each action (e.g. command line executions and arguments, menu entries, keyboard shortcuts, screenshots). The use case is the following:
\begin{enumerate}
\item Start synchronization server on localhost.
\item Create account on synchronization server.
\item Create two VFS disks (on the same machine) and link them to the new account.
\item Import a directory (recursively) from the host file system into Disk 1.
\item Dispose Disk 1 after the synchronization finished.
\item Export the directory (recursively) from Disk 2 into the host file system.
\item Stop synchronization server.
\end{enumerate}
}


\end{document}
